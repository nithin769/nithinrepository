%iffalse
\let\negmedspace\undefined
\let\negthickspace\undefined
\documentclass[journal,12pt,twocolumn]{IEEEtran}
\usepackage{cite}
\usepackage{amsmath,amssymb,amsfonts,amsthm}
\usepackage{algorithmic}
\usepackage{graphicx}
\usepackage{textcomp}
\usepackage{xcolor}
\usepackage{txfonts}
\usepackage{listings}
\usepackage{enumitem}
\usepackage{mathtools}
\usepackage{gensymb}
\usepackage{comment}
\usepackage[breaklinks=true]{hyperref}
\usepackage{tkz-euclide} 
\usepackage{listings}
\usepackage{gvv}   
\def\inputGnumericTable{}                                 
\usepackage[latin1]{inputenc}                                
\usepackage{color}                                            
\usepackage{array}                                            
\usepackage{longtable}                                       
\usepackage{calc}                                             
\usepackage{multirow}                                         
\usepackage{hhline}                                           
\usepackage{ifthen}                                           
\usepackage{lscape}
\usepackage{tabularx}
\usepackage{array}
\usepackage{float}

\newtheorem{theorem}{Theorem}[section]
\newtheorem{problem}{Problem}
\newtheorem{proposition}{Proposition}[section]
\newtheorem{lemma}{Lemma}[section]
\newtheorem{corollary}[theorem]{Corollary}
\newtheorem{example}{Example}[section]
\newtheorem{definition}[problem]{Definition}
\newcommand{\BEQA}{\begin{eqnarray}}
\newcommand{\EEQA}{\end{eqnarray}}
\newcommand{\define}{\stackrel{\triangle}{=}}
\theoremstyle{remark}
\newtheorem{rem}{Remark}

% Marks the beginning of the document
\begin{document}
\bibliographystyle{IEEEtran}

\title{Assignment-1\\CHAPETR-11\\Limits, Continuity and Differentiability}
\author{EE24BTECH11048-NITHIN.K} 
\maketitle
\newpage
\bigskip

\renewcommand{\thefigure}{\theenumi}
\renewcommand{\thetable}{\theenumi}

\fontsize{14}{16}\selectfont
\textcolor{black}{C : MCQs with One Correct Answer \\}


\begin{enumerate} 
\item Let g(x)=$\frac{(x-1)^n}{logcos^m(x-1)}$ ; $0<x<2$, m and n are integers, $m\neq0$, $n>0$, and let p be the left hand derivative of $|x-1|$ at $x=1$. If $\lim_{x \to 1^+}{g(x)=p}$, then \hfill{(2008)} \\  

\begin{enumerate}[label=\alph*)]
    \item $n=1,m=1$
    \item $n=1,m=-1$
    \item $n=2,m=2$
    \item $n>2,m=n$
\end{enumerate}

\item If $\lim_{x \to 0}[1+xln(1+b^2)]^\frac{1}{x}$ = $2bsin^2\theta$, $b>0$ and $\theta \in (-\pi,\pi]$, then the value of $\theta$ is 
\hfill{(2011)} \\

\begin{enumerate}[label=\alph*)]
    \item $\pm\frac{\pi}{4}$
    \item $\pm\frac{\pi}{3}$
    \item $\pm\frac{\pi}{6}$
    \item $\pm\frac{\pi}{2}$
\end{enumerate} 

\item If $\lim_{ x \to \infty}\left(\frac{x^2+x+1}{x+1}-ax-b\right) = 4$, then \hfill{(2012)} \\


\begin{enumerate}[label=\alph*)]
    \item $a=1,b=4$
    \item $a=1,b=-4$
    \item $a=2,b=-3$
    \item $a=2,b=3$
\end{enumerate}


\item 
Let $f(x) =\begin{cases} x^2|cos \frac{\pi}{x}| & \text{, } x \neq 0 \\ 0 & \text{, } x = 0 
\end{cases}
$, x $\in$ R then f is
\hfill{(2012)} \\
\begin{enumerate}[label=\alph*)]
    \item differentiable both at $x=0$ and at $x=2$
    \item differentiable at x=0 but not differentiable at x=2
    \item not differentiable at x=0 but differentiable at x=2
    \item differentiable neither at x=0 nor at x=2
\end{enumerate}

\item 
Let $\alpha(a)$ and $\beta(a)$ be the roots of the equation $(\sqrt[3]{1+a}-1)x^2+(\sqrt[2]{1+a}-1)x+(\sqrt[6]{1+a}-1)=0$ where $a>-1$. then $\lim_{a \to 0^+}{\alpha(a)}$ and $\lim_{a \to 0^+}{\beta(a)}$ are
\hfill{(2012)} \\

\begin{enumerate}[label=\alph*)]
    \item $-\frac{5}{2}$ and 1
    \item $-\frac{1}{2}$ and 1
    \item $-\frac{7}{2}$ and 2
    \item $-\frac{9}{2}$ and 3
\end{enumerate}


\fontsize{14}{16}\selectfont
\textcolor{black}{D : MCQs with One or More than One Correct \\ }

\item If $x+|y|=2y$, then y as a function of x is
\hfill{(1984-3marks)} \\
\begin{enumerate}[label=\alph*)]
    \item defined for all real x
    \item continuous at $x=0$
    \item differentiable for all x
    \item such that $\frac{dy}{dx}=\frac{1}{3}$ for$x<0$
\end{enumerate}

\item If $f(x)=x(\sqrt{x}-\sqrt{x+1})$, then- \\
\hfill{(1985-2marks)} \\
\begin{enumerate}[label=\alph*)]
    \item f(x) is continuous but not differentiable at $x=0$
    \item f(x) is differentiable at $x=0$
    \item f(x) is not differentiable at $x=0$
    \item none of these
\end{enumerate}

\item The function $f(x)=1+|sinx|$ is \\
\hfill{(1986-2marks)} 
\begin{enumerate}[label=\alph*)]
    \item continuous nowhere
    \item continuous everywhere
    \item differentiable nowhere 
    \item not differentiable at $x=0$
    \item not differentiable at infinite number of points
\end{enumerate} 

\item Let [x] denote the greatest integer less than or equal to x. If $f(x)=[xsin \pi x]$, then f(x) is \\
\hfill{(1986-2marks)} \\
\begin{enumerate}[label=\alph*)]
    \item continuous at $x=0$
    \item continuous in (-1,0)
    \item differentiable at $x=1$ 
    \item differentiable in (-1,1) 
    \item none of these 
\end{enumerate}


\item The set of all points where the function $f(x)=\frac{x}{(1+|x|)}$ is differentiable, is \\
\hfill{(1987-2marks)} \\

\begin{enumerate}
    \item \((- \infty, \infty)\)
    \item \([0, \infty)\)
    \item \((- \infty,0)\) $\bigcup$ \((0, \infty)\)
    \item (0, $\infty$)
    \item None
\end{enumerate}

\item The function \\ $f(x)=\begin{cases}|x-3|,\ \ \ \ \ \ \ \ \ $x$ \geq $1$ \\
    \frac{x^2}{4}-\frac{3x}{2}+\frac{13}{4},\ \ x<1 
\end{cases}$,is \\ \hfill{(1988-2marks)} \\
\begin{enumerate}
    \item continuous at x=1
    \item differentiable at x=1
    \item continuous at x=3
    \item differentiable at x=3 \\ \\ \\ \\ \\
\end{enumerate}

\item If $f(x)$=$\frac{1}{2}x-1$, then on the interval $[0,\pi]$ \hfill{(1989-2marks)} \\
\begin{enumerate}
    \item tan[f(x)] and $\frac{1}{f(x)}$ are both continuous
    \item tan[f(x)] and $\frac{1}{f(x)}$ are both discontinuous
    \item tan[f(x)] and $f^{-1}x$ are both continuous
    \item tan[f(x)] is continuous but $\frac{1}{f(x)}$ is not \\
\end{enumerate}


\item The value of $\lim_{x\to0}{\frac{\sqrt{\frac{1}{2}(1-cos2x)}}{x}}$ \\
\hfill{(1991-2marks)} \\
\begin{enumerate}
    \item 1
    \item -1
    \item 0
    \item none of these
\end{enumerate}

\item The following functions are continuous on (0,$\pi$) \hfill{(1991-2marks)} \\
\begin{enumerate}
    \item tanx
    \item $\int_{0}^{x}tsin\frac{1}{t}dt$ 
    \item $\begin{cases} 1,\ \ \ \ \ \ \ \ \ \ 0<x\leq\frac{3\pi}{4} \\
                2sin\frac{2}{9}x,\ \ \ \frac{3\pi}{4}<x<\pi
    
\end{cases}$
    \item $\begin{cases} xsinx,\ \ \ \ \ \ \ \ \ 0<x\leq\frac{\pi}{2} \\
                  \frac{\pi}{2}sin(\pi+x),\ \ \frac{\pi}{2}<x<\pi
\end{cases}$
\end{enumerate}

\item Let $f(x)=\begin{cases} 0,\ \ \ \ x<0 \\
                             x^2,\ \ \ x\leq0
\end{cases}$ then for all x \hfill{(1994)} \\

\begin{enumerate}
    \item $f^|$ is differentiable
    \item f is differentiable
    \item $f^|$ is continuous 
    \item f is continuous
\end{enumerate}


\end{enumerate}

\end{document}
